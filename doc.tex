\documentclass{article}
\usepackage{array}
\usepackage{multirow}
\usepackage[left=15mm,right=15mm]{geometry}

\newcolumntype{M}[1]{>{\centering\arraybackslash}p{#1}}

\author{Johan Martinson\\Team Debit}
\begin{document}

\title{Summary of benefits and use cases of Feature Flags}
\maketitle


\subsection*{Use cases for the finance domain}

\begin{table}[h!]
    \begin{center}
        \caption{Feature flag use-cases}
        \label{tab:useCases}
        \renewcommand{\arraystretch}{1.5} % Increase the height of rows that are not part of a multirow
        \begin{tabular}{m{.25\linewidth}lm{.5\linewidth}} % Alignments: 1st column left, 2nd middle and 3rd right, with vertical lines in between
            \textbf{Use Case} & \textbf{Feature Flag type} & \textbf{User Story}                                                                                                                         \\
            \multirow{2}{\linewidth}{Consistently deploy code to production.} & \multirow{2}{*}{\centering Non-static} & As a Centiro developer I want to consistently push code to production, without having effect until I enable the code through a feature flag. \\
             & & As a Centiro developer I want to remove feature flags once the feature is tested and implemented as standard (a mandatory feature.)                                                                                                                                 \\
             \hline
             \multirow{1}{\linewidth}{Use of optional features.} & \multirow{1}{*}{\centering Static} & As a Centiro developer I want to have certain parts of code enabled for certain tenants especially when the feature is customer specific. \\
        \end{tabular}
    \end{center}
\end{table}

\subsection*{Benefits}
\begin{itemize}
    \item
\end{itemize}



\end{document}